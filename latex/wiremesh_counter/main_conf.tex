\documentclass[conference]{IEEEtran}

\usepackage[utf8]{inputenc}
%\usepackage{graphicx}
\usepackage{caption}
\usepackage{subcaption}
\usepackage{subfiles}

% *** GRAPHICS RELATED PACKAGES ***
%
\ifCLASSINFOpdf
  \usepackage[pdftex]{graphicx}
  % declare the path(s) where your graphic files are
  % \graphicspath{{../pdf/}{../jpeg/}}
  % and their extensions so you won't have to specify these with
  % every instance of \includegraphics
  % \DeclareGraphicsExtensions{.pdf,.jpeg,.png}
\else
  % or other class option (dvipsone, dvipdf, if not using dvips). graphicx
  % will default to the driver specified in the system graphics.cfg if no
  % driver is specified.
  % \usepackage[dvips]{graphicx}
  % declare the path(s) where your graphic files are
  % \graphicspath{{../eps/}}
  % and their extensions so you won't have to specify these with
  % every instance of \includegraphics
  % \DeclareGraphicsExtensions{.eps}
\fi



% correct bad hyphenation here
\hyphenation{op-tical net-works semi-conduc-tor}


\begin{document}
%
% paper title
% can use linebreaks \\ within to get better formatting as desired
\title{Diseño e Implementación de un Sistema de Medida de Longitud de Hebras Metálicas de Bajo Costo para la Reducción de Pérdidas en una Planta de Fabricación de Mallas Metálicas}


% author names and affiliations
% use a multiple column layout for up to three different
% affiliations
\author{\IEEEauthorblockN{Frank A. Moreno Vera}
\IEEEauthorblockA{Departamento de automatización\\
GeckoTronics S.A.C.\\
Lima, Perú\\
Email: frank.moreno@geckotronics.pe}
\and

\IEEEauthorblockN{Jhonatan D. Paucará Prado}
\IEEEauthorblockA{Departamento de automatización\\
GeckoTronics S.A.C.\\
Lima, Perú\\
Email: jhonatan.paucara@gmail.com}
}

% use for special paper notices
%\IEEEspecialpapernotice{(Invited Paper)}


% make the title area
\maketitle


\begin{abstract}
%\boldmath
The abstract goes here.
\end{abstract}
% IEEEtran.cls defaults to using nonbold math in the Abstract.
% This preserves the distinction between vectors and scalars. However,
% if the conference you are submitting to favors bold math in the abstract,
% then you can use LaTeX's standard command \boldmath at the very start
% of the abstract to achieve this. Many IEEE journals/conferences frown on
% math in the abstract anyway.

% no keywords

% For peer review papers, you can put extra information on the cover
% page as needed:
% \ifCLASSOPTIONpeerreview
% \begin{center} \bfseries EDICS Category: 3-BBND \end{center}
% \fi
%
% For peerreview papers, this IEEEtran command inserts a page break and
% creates the second title. It will be ignored for other modes.
\IEEEpeerreviewmaketitle

\section{Introduction}
Las plantas de fabricación de mallas metálicas usan máquinas de 
confección de hebras metálicas de diferentes grosores y 
longitudes. Estas máquinas no suelen poseer sistema 
automatizado, solo están los controles On/Off manual, por lo 
cual la medición de longitud de cada hebra es manual, lenta y con 
grandes pérdidas. El proceso de automatización
consiste en reducir al máximo el error de medición para así 
reducir la merma y reducir costos de producción. Así mismo, este 
sistema será fabricado con materiales de bajo costo y de fácil 
obtención en el mercado local.

\section{Sistema estimador de longitud de hebra metálica}

\subfile{sistema_estimador_hebra_metalica}

\section{Caja eléctrica a modificar}

\subfile{caja_electrica}

\section{Interfaz de configuración y supervisión}

\subfile{interfaz_conf_supervision}

\section{Conclusiones}

The conclusion goes here.

% conference papers do not normally have an appendix

% use section* for acknowledgement

\section{Desarrollo y alcances futuros}

\subfile{desarrollo_y_alcance_futuro}

\end{document}
